\documentclass{article}
\usepackage{graphicx, geometry}
\geometry{margin=1in}
\pagenumbering{gobble}

\begin{document}
\title{NASA: Optimization of Airport Surface Planning and Scheduling}
\author{Anshu Rajendra, Heron Yang, Ritwik Rajendra\\
	\texttt{anshur@andrew.cmu.edu, \{heronyang, ritwikr\}@cmu.edu}\\
	\and
	Sponsor Mentor: Robert A. Morris\\
	NASA Ames Research Center\\
	\texttt{robert.a.morris@nasa.gov}\\
	\and
	Advisor: Corina Pasareanu\\
    Carnegie Mellon University Silicon Valley\\
	\texttt{corina.pasareanu@west.cmu.edu}
}

\maketitle

\section{Abstract}
    This paper introduces planning and scheduling in the context of real world uncertainty. It demonstrates the use of probabilistic modeling to develop and implement scalable optimization algorithms to improve surface operations at large airports. As part of practicum work, we created a generic airport simulation tool (ASSET2) that is easily extensible to different scenarios and airports and improves upon the basic ASSET simulator used in this context before. We also explored different auto-scheduling methods to compare with current FCFS method. We created an uncertainty aware scheduler which produces roust schedules with simulated uncertainty. Lastly, we built a uncertainty module to model real word uncertainty and evaluate the robustness of scheduler in light of different scenarios. 

\section{Conclusion}

In this project, we've successfully built a system, ASSET2, to simulate surface movements of an airport using different scheduling algorithm with different control variables and uncertainties. The system is composed by three main modules: simulator, scheduler, and uncertainty module. Simulator is in charge of maintaining the airport state and coordinates every components in the system. Scheduler is be triggered by the simulator and offers the best schedule based on a given airport state. Uncertainty module is injected in the simulation and will create chaos for users to obverse the performance of the scheduler under uncertainty.

We ran two experiments using ASSET2 to see if we can plot the figures that we expected. In the first experiment, we measure the effects of schedule tightness on the amount of run-time conflicts for varying degrees of uncertainty. We found that the result is different because large tightness values resulted aircrafts being delayed too much. In the second experiment, we fix the level of tightness and measure the effect of different levels of re-planning on the amount of run-time conflicts, and we found the result confirm our hypothesis.

% that's all folks
\end{document}
